% !TEX encoding = UTF-8 Unicode

\documentclass{article}
\usepackage[french]{babel}
\author{Louis DESVERNOIS}
\title{Procédure d'installation et de déploiement pour la SAÉ23.}
% \date{9 Juin 2022}
\usepackage{amsmath}
\usepackage[margin=1in]{geometry}
\usepackage{graphicx}
\usepackage{subcaption}
\usepackage{listings}
\usepackage{minted}
\usepackage{newfloat}
%\setcounter{tocdepth}{1} % Show sections
%\setcounter{tocdepth}{2} % + sections
%\setcounter{tocdepth}{3} % + subsections
%\setcounter{tocdepth}{4} % + paragraphs
%\setcounter{tocdepth}{5} % + subparagraphs
%\renewcommand{\contentsname}{Tables des matières} % Change le nom de la ToC
%\renewcommand{\listfigurename}{Liste des figures}
\usepackage{minted,xcolor,chngcntr}

% Style Minted
\usemintedstyle{default}
\definecolor{codebg}{rgb}{0.96,0.96,0.96}
\newminted{python}{bgcolor=codebg,
                    linenos=true,
                    frame=lines,
                    numbersep=5pt,
                    fontsize=\footnotesize}
\renewcommand{\listoflistingscaption}{Table des codes}
\renewcommand{\listingscaption}{Code}
% \renewcommand{\bibname}{Bibliographie} % Change le nom des références (bibname pour les classes book et report)

\begin{document}
    \maketitle
    \tableofcontents
    \listoffigures
    \listoflistings
    \newpage
    \section{Machine virtuelle et base de données}
        Notre solution se base sur une machine virtuelle utilisant la dernière version de Debian 11. La technologie utilisée pour créer cette machine virtuelle n'a peu d'importance, tant que celle-ci est accessible (eg: carte réseau en mode bridge).
            \subsection{Installations des dépendances}
            Après l'installation d'un système minimal Debian 11, nous avons besoin d'installer les différents composant nécessaire au déploiement d'un serveur MariaDB ainsi qu'un serveur HTTP nginx.
            \begin{listing}[H]
                \begin{minted}[breaklines]{bash}
        apt install git mariadb-server nginx python3 python3-pip python3-venv python3-dev libmariadb-dev ufw -y               
                \end{minted}
                \caption{Installation des dépendance}
                \label{lst:deps-install}
            \end{listing}
            \subsection{Configuration de la base de données}
            En installant le paquet \verb|mariadb-server|, le gestionnaire de paquets apt a déjà automatique activé le service. Il nous reste donc qu'a configurer le serveur.
            \begin{listing}[H]
                \begin{minted}[breaklines]{sql}
        mysql -sfu root <<EOS
        UPDATE mysql.user SET Password=PASSWORD('toto') WHERE User='root';
        DELETE FROM mysql.user WHERE User='';
        DROP DATABASE IF EXISTS test;
        DELETE FROM mysql.db WHERE Db='test' OR Db='test\\_%';
        FLUSH PRIVILEGES;
        CREATE USER 'toto'@'localhost';
        EOS
                \end{minted}
                \caption{Configuration initiale du serveur MariaDB}
            \end{listing}
            Pour la configuration nous utilisons la commande \verb|mysql -sfu root| pour nous connecter à la base de données et ignorer les erreurs. Pour commencer, nous changeons le mot de passe de l'utilisateur "root", nous supprimons tous les utilisateurs anonymes, nous supprimons la base de données "test" si elle existe, puis nous créons l'utilisateur "toto", qui permettera à Django d'accéder à la base de données.
    \section{Environnement virtuel Python}
        Pour faire fonctionner Django, nous allons avoir besoin d'un environnement virtuel (venv) pour installer Django et ses dépendances sans les installer pour tout le système. Travailler avec des venv permet de garantir que paquets installés soient toujours les mêmes. 
            \subsection{Création de l'environnement}
            Le module \verb|venv| de Python nous permet de créer ces environnements facilement avec la commande \verb|python -m venv .venv|. En supposant que l'on utilise le shell bash, nous pouvons ensuite activer cet environnement grâce à la commande \verb|source|.
            \begin{listing}[H]
                \begin{minted}{bash}
        python -m venv .venv
        source .venv/bin/activate
                \end{minted}
                \caption{Création et activation du venv}
                \label{lst:creation-venv}
            \end{listing}

            \subsection{Installation des paquets}

\end{document}